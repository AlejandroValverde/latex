\section*{Abstract}

This thesis presents a novel purely passive mechanism of twist morphing for application on airframe structures, such as wings. During certain flight conditions such as gust encountering, the aircraft may experience critical loads that threaten the structural integrity. To counteract this situation, a rapid modification of the lift distribution to achieve a reduction in the aerodynamic loads is needed. The proposed method aims to control the bending-twist coupling of the wing-box and in based in a variable-stiffness adaptive spar implementation. Through modifications in the effective shear modulus in this adaptive spar, variations in the torsional stiffness of the wing-box are induced. The mechanism uses intentionally originated elastic instabilities in a structure of chiral elements to induce this sudden reductions in the adaptive spar stiffness. Therefore, snap-buckling instabilities are used to achieve the desired sectional twist of the wing-box. An analytical model of the wing-box is developed to provide information related to the changes in mechanical properties through modifications in shear modulus on the variable-stiffness spar. A computational model of the whole assembly is built to provide insight of the buckling phenomena and the nonlinear response in twist of the structure. An extensive analysis of the influence of each of the main parameters in the buckling appearance and evolution, and in final twist response is presented. Numeric results show that snap-buckling instabilities indeed provide an effective way of changing the torsional stiffness of a wing-box and therefore affecting the twist and the lift distribution over the wing. Also, results show that parameters such as the wing-box thickness or the ligament eccentricity provide tailorability capabilities over the onset of triggering buckling at a prescribed level of external loading. 

%A reduction in the penalties asso-ciated to the added structural mass required to withstand rare load scenarios by means of load alleviation control is highly desirable, particularly for efficient light-weight engineering systems, such as aircraft and wind turbine blades.

% \vspace{40mm}
% \noindent
% \textbf{Keywords: }chiral, morphing aircraft, variable-stiffness, bending-twist coupling

\vfill
\normalsize
\noindent
This project was realized within the Laboratory of Composite Materials and Adaptive Structures of the Eidgen\"ossische Technische Hochschule Z\"urich (Swiss Federal Institute of Technology Zurich), under the supervision of Prof. Dr. P. Ermanni and the advisory of F. Runkel, K. Dominic and U. Fasel.