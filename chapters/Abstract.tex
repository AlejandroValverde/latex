\section*{Abstract}

This thesis presents a novel purely passive mechanism of twist morphing for application on aerospace structures, such as wings. During certain flight conditions such as gust encountering, the aircraft may experience critical loads that threaten the structural integrity. To counteract this situation, a time-bounded modification of the lift distribution to achieve a reduction in the aerodynamic load is needed. The proposed method aims to control the bending-twist coupling of the wing-box affecting its torsional stiffness. The design of this wing-box incorporates a variable-stiffness adaptive spar implementation. This element is comprised of a lattice of chiral elements that undergo elastic instabilities on its ligaments under certain load, originating a sudden reduction of the shear modulus in the adaptive spar. The modification of the effective shear modulus in this element provokes the wing-box shear center shifting, a consequent modification of the torsional stiffness and, ultimately, a buckling-induced sectional twist in the wing-box. The mechanism therefore consist on a highly nonlinear variation of the wing twist passively triggered by the onset of buckling on the chiral structure. An analytical model of the wing-box is developed to provide information related to the changes in mechanical properties through modifications in shear modulus on the variable-stiffness spar. A fully parametrized computational model of the whole assembly is built to provide insight of the buckling phenomena and the nonlinear twist response of the structure. An extensive analysis of the influence of each of the main parameters in the buckling appearance and evolution, and in final twist response is also presented. Numeric results show that the snap-buckling instabilities indeed provide an effective method of changing the torsional stiffness of a wing-box and therefore affecting the twist and the lift distribution over the wing. Also, results show that parameters such as the wing-box thickness or the ligament eccentricity provide tailorability capabilities over the mechanism activation and the response evolution.

%A reduction in the penalties asso-ciated to the added structural mass required to withstand rare load scenarios by means of load alleviation control is highly desirable, particularly for efficient light-weight engineering systems, such as aircraft and wind turbine blades.

% \vspace{40mm}
% \noindent
% \textbf{Keywords: }chiral, morphing aircraft, variable-stiffness, bending-twist coupling

\vfill
\normalsize
\noindent
This project was realized within the Laboratory of Composite Materials and Adaptive Structures of the Eidgen\"ossische Technische Hochschule Z\"urich (Swiss Federal Institute of Technology Zurich), under the supervision of Prof. Dr. P. Ermanni and the advisory of F. Runkel, K. Dominic and U. Fasel.