\section{Micromechanics}
The temperature and conversion calculated in the cure simulation can be used to define the lamina properties and the strain for the stress analysis.
\subsection{Lamina properties}
\subsubsection{Resin properties}
At first, a model for the elastic modulus has to be applied. The model of Bogetti and Gillespie \cite{bogetti_process-induced_1992} is a cure dependent model. To proof that the resin properties follow these behaviour, several tests are made with the technique of the Dynamic Mechanical Thermal Analysis (DMTA \cite{wetton1986dynamic}). The resin is cured to a controlled degree of conversion by using different cure temperatures. The values of the cure conversion of the parts are $0.73$, $0.81$, $0.86$ and $1$. After that, the  elastic properties of the parts are measured with a three point bending test from $20°C$ to $160°C$ (DMTA). The results show different values of the maximum modulus at different cure conversions. An estimation for the maximum resin modulus has to be made. It has his maximum value around the value of $\alpha=0.73$. After that it has a linear decreasing behaviour until the full conversion.  and a linear decrease of by higher temperature until the reach of the glass 