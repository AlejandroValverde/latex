\section*{Zusammenfassung}

Diese Masterarbeit stellt einen neuen, rein passiven, Mechanismus für die Formveränderung von Flugzeugstrukturen, wie beispielsweise Flügelstrukturen, dar. Während gewissen Flugbedingungen, wie zum Beispiel während des Fluges in Turbulenzen, kann das Flugzeug kritische Lasten erfahren, welche die Struktur gefährden. Solch kritische Bedingungen sollen durch die Anpassung der Flügelgeometrie, welche die Aerodynamische lasten reduziert, vermieden werden. Das Ziel der vorgeschlagenen Methode ist es, die Kopplung zwischen Durchbiegung und Verdrehung des Flügels mithilfe eines Holmes mit variabler Steifigkeit zu kontrollieren. Der Holm besteht aus einem Netz von chiralen Elementen. Das gewählte Design erreicht eine veränderbare Schubsteifigkeit durch eine lokale elastische Unstabilität im chiralen Netz. Die Veränderung der effektiven Schubsteifigkeit in den Elementen löst eine Verschiebung des Schubmittelpunkts des Flügelkastens und somit eine Modifikation der Torsionssteifigkeit aus. Als dessen Resultats erfolgt eine durch Beulen erzeugte regionale Verdrehung des Flügelkastens. Der Mechanismus besteht aus nichtlinearen Variationen der Verdrehung des Flügelkastens, passiv ausgelöst durch das Beulen der chiralen Struktur. Zunächst wurde anhand einer vereinfachten Struktur die Machbarkeit des Konzeptes aufgezeigt. Ein analytisches Model des Flügelkastens wurde entwickelt, um den Effekt der Veränderung des Schubmoduls auf die variable Steifigkeit des Holmes aufzuzeigen.  Ein komplett parametrisiertes Computermodel des Flügelkastens wurde aufgebaut, um Erkenntnisse über das Beulphänomen und die nichtlineare Antwort der Struktur auf Verdrehen zu gewinnen. Eine umfassende Analyse der Einflusse jedes Parameters im Beulprozess und deren Auswirkung auf den Flügelkasten wurde durchgeführt. Numerische Resultate zeigen, dass Beulinstabilitäten eine effektive Methode zur Veränderung der Torsionsteifigkeit darstellen, womit die Torsions- und Auftriebsverteilung verändert werden können. Zusätzlich zeigen die Resultate, dass sich durch Anpassung gewisser Parameter, wie die Flügelkastendicke oder die Stegexzentrizität, das Eintreten und die Veränderung der Torsionsantwort steuern lassen. Beginnend mit einem vereinfachten Flügelkasten wurde das Konzept auf eine reale Flügelstruktur erweitert. Ein Modell des Flügels mit einer nachgiebigen Flügelkastenstruktur wurde implementiert und mit einer statischen aerolastischen Analyse gekoppelt. Vorläufige Resultate zeigen, dass für einen Flügel mit einer Spannweite von 3 m und einer Fluggeschwindigkeit von 30 m/s Veränderungen der Verdrehung an der Flügelspitze von bis zu 4 Grad erreicht werden können, welche somit die Durchführbarkeit des Konzepts beweisen.

%A reduction in the penalties asso-ciated to the added structural mass required to withstand rare load scenarios by means of load alleviation control is highly desirable, particularly for efficient light-weight engineering systems, such as aircraft and wind turbine blades.
%
% \vspace{40mm}
% \noindent
% \textbf{Keywords: }chiral, morphing aircraft, variable-stiffness, bending-twist coupling