\section*{Zusammenfassung}

Diese Masterarbeit stellt einen neuen, rein passiven, Mechanismus f\"ur die Formver\"anderung von Flug\-zeu\-gstrukturen, wie beispielsweise Fl\"ugelstrukturen, dar. W\"ahrend gewissen Flugbedingungen, wie zum Beispiel w\"ahrend des Fluges in Turbulenzen, kann das Flugzeug kritische Lasten erfahren, welche die Struktur gef\"ahrden. Solch kritische Bedingungen sollen durch die Anpassung der Fl\"ugelgeometrie, welche die Aerodynamische lasten reduziert, vermieden werden. Das Ziel der vorgeschlagenen Methode ist es, die Kopplung zwischen Durchbiegung und Verdrehung des Fl\"ugels mithilfe eines Holmes mit variabler Steifigkeit zu kontrollieren. Der Holm besteht aus einem Netz von chiralen Elementen. Das gew\"ahlte Design erreicht eine ver\"anderbare Schubsteifigkeit durch eine lokale elastische Unstabilit\"at im chiralen Netz. Die Ver\"anderung der effektiven Schubsteifigkeit in den Elementen l\"ost eine Verschiebung des Schubmittelpunkts des Fl\"ugelkastens und somit eine Modifikation der Torsionssteifigkeit aus. Als dessen Resultats erfolgt eine durch Beulen erzeugte regionale Verdrehung des Fl\"ugelkastens. Der Mechanismus besteht aus nichtlinearen Variationen der Verdrehung des Fl\"ugelkastens, passiv ausgel\"ost durch das Beulen der chiralen Struktur. Zun\"achst wurde anhand einer vereinfachten Struktur die Machbarkeit des Konzeptes aufgezeigt. Ein analytisches Model des Fl\"ugelkastens wurde entwickelt, um den Effekt der Ver\"anderung des Schubmoduls auf die variable Steifigkeit des Holmes aufzuzeigen.  Ein komplett parametrisiertes Computermodel des Fl\"ugelkastens wurde aufgebaut, um Erkenntnisse \"uber das Beulph\"anomen und die nichtlineare Antwort der Struktur auf Verdrehen zu gewinnen. Eine umfassende Analyse der Einflusse jedes Parameters im Beulprozess und deren Auswirkung auf den Fl\"ugelkasten wurde durchgef\"uhrt. Numerische Resultate zeigen, dass Beulinstabilit\"aten eine effektive Methode zur Ver\"anderung der Torsionsteifigkeit darstellen, womit die Torsions- und Auftriebsverteilung ver\"andert werden k\"onnen. Zus\"atzlich zeigen die Resultate, dass sich durch Anpassung gewisser Parameter, wie die Fl\"ugelkastendicke oder die Stegexzentrizit\"at, das Eintreten und die Ver\"anderung der Torsionsantwort steuern lassen. Beginnend mit einem vereinfachten Fl\"ugelkasten wurde das Konzept auf eine reale Fl\"ugelstruktur erweitert. Ein Modell des Fl\"ugels mit einer nachgiebigen Fl\"ugelkastenstruktur wurde implementiert und mit einer statischen aerolastischen Analyse gekoppelt. Vorl\"aufige Resultate zeigen, dass f\"ur einen Fl\"ugel mit einer Spannweite von 3 m und einer Fluggeschwindigkeit von 30 m/s Ver\"anderungen der Verdrehung an der Fl\"ugelspitze von bis zu 4 Grad erreicht werden k\"onnen, welche somit die Durchf\"uhrbarkeit des Konzepts beweisen.

%A reduction in the penalties asso-ciated to the added structural mass required to withstand rare load scenarios by means of load alleviation control is highly desirable, particularly for efficient light-weight engineering systems, such as aircraft and wind turbine blades.
%
% \vspace{40mm}
% \noindent
% \textbf{Keywords: }chiral, morphing aircraft, variable-stiffness, bending-twist coupling