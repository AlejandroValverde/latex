\chapter{Model analysis} \label{chap:Results_model}
%
% Overview:
% - Analytical model: Parametric study
% - Connection lattice - wing-box skin
% - Mesh: mesh problems
% - Artifical damping factor
% - Inner ribs
  
  %Intro
  In the present chapter the general characterization of the model is presented. The first section includes an analysis of the analytical model. For this, a variation of the beam geometric parameters and the stiffness ratio $E_1/E_2$ is performed. Firstly, the predicted evolution of the bending-twist coupling as a function of the stiffness ratio $E_1/E_2$ is presented. Secondly, the results obtained from the effect of the beam geometric parameters on the section mechanical properties are introduced. This will provide insight of how the different design parameters affect the final mechanical properties of the beam.

  In the second section, the computational model is analysed in order to obtain the most suitable configuration to show the response expected from the concept proposed. In particular, discussions over the methods to model the rigid body behavior of the lattice nodes, the load introduction method, the mesh particularities, the ribs inclusion and the nonlinearities the response are presented. 

\section{Analytical model analysis} \label{sec:analyticalParametricStudy_results_model}

  The analytical model of the wing-box was already presented in the Section \ref{sec:analytical_Model}. For the results that are presented in the subsections below, the parameters already defined in the mentioned section are assigned to the nominal values presented in Table \ref{tab:parameters_analytical}. These are taken from a similar analytical approach to the problem of a wing-box with a variable-stiffness web, presented in \cite{Raither2013a}. By doing this, verification of the results becomes possible.

  \begin{table}[!htpb]
  \centering
  \begin{tabular}{|l|lll|}
  \hline
  \textbf{Parameter} & \multicolumn{1}{l|}{\textbf{Symbol}} & \multicolumn{1}{l|}{\textbf{Units}} & \textbf{Nominal value} \\ \hline \hline
  {\textbf{Dimensions}} &  &  &  \\ \hline
  Height of the cross section & \multicolumn{1}{l|}{$H$} & \multicolumn{1}{l|}{mm} & 200 \\ \hline
  Wing-box length & \multicolumn{1}{l|}{$L$} & \multicolumn{1}{l|}{mm} & 800 \\ \hline
  Width of the cross section & \multicolumn{1}{l|}{$B$} & \multicolumn{1}{l|}{mm} & 80 \\ \hline
  Wing-box wall thickness & \multicolumn{1}{l|}{$t_1, t_2$} & \multicolumn{1}{l|}{mm} & 1 \\ \hline \hline
  {\textbf{Material (Aluminum)}} &  &  &  \\ \hline
  Young's modulus & \multicolumn{1}{l|}{$E_1, E_2$} & \multicolumn{1}{l|}{N/mm$^2$} & 69000 \\ \hline
  Shear modulus & \multicolumn{1}{l|}{$G_1, G_2$} & \multicolumn{1}{l|}{N/mm$^2$} & 26000 \\ \hline
  \end{tabular}
  \caption[Nominal value of the parameters used for the analytical model]{Nominal value of the parameters used for the analytical model. The mechanical properties of the material used correspond to standard aluminum.}
  \label{tab:parameters_analytical}
  \end{table}

  \clearpage
  \subsection{Bending and twisting coupling results and discussion} \label{subsec:bendingTwistCoupling_results_model}
  
    Here, the influence of the the stiffness ratio $E_1/E_2$ on the coupling between the bending and the twist deformations of the beam is presented. This is graphically shown in Figure \ref{fig:twist-E1overE2} for a number of simulations performed for a load $Q_z$ equal to 2000 N. In the plot, the twist deformation at the tip is shown with the solid line and it is represented using the variable $\phi_{\mathrm{tip}}/Q$ while the bend deformation is shown with the dashed line and it is also represented as $w_{\mathrm{tip}}/Q$.

    \begin{figure}[!htpb] %twist and bend versus E1/E2
      \centering
      \includegraphics[width=0.8 \textwidth]{../../analytical/figures/twist-E1overE2}
      \caption[Influence of the stiffness ratio on the wing-box tip twist and bend]{Influence of the stiffness ratio $E_1/E_2$ on the wing-box tip twist $\phi_{\mathrm{tip}}$ and bend $w_{\mathrm{tip}}$. The solid line is used to represent the twist $\phi_{\mathrm{tip}}$ while the dashed line represents the bend $w_{\mathrm{tip}}$. The two displacements are divided by the magnitude of the applied force $Q_z$.}\label{fig:twist-E1overE2}
    \end{figure}

    In Figure \ref{fig:twist-E1overE2}, the stiffness ratio $E_1/E_2$ is modified over a wide range such that $E_1/E_2 \in [10^{0}, 10^{3}]$. It can be seen that the variable $\phi_{\mathrm{tip}}/Q_z$ is consequently increased by various orders of magnitude from $\phi_{\mathrm{tip}}/Q_z = 10^{-4}$ deg/N to $\phi_{\mathrm{tip}}/Q_z = 10^{-1}$ deg/N. However the variation in bend represented with the variable $w_{\mathrm{tip}}/Q_z$ is negligible in comparison. In can be seen that the bend displacement quickly increases for values of $E_1/E_2 > 1$ and it gets to an asymptote for values $E_1/E_2 \gg 1$. These results therefore show how the twist displacement, which gives information regarding the torsional stiffness of the structure, is much more affected by variations of the stiffness ratio $E_1/E_2$ than the bending stiffness is. Consequently, it is not expected to see large increments in bend displacement in comparison with the twist when the variable-stiffness mechanism is activated. This results are in compliance with those presented in \cite{Raither2013a}.

    For the proposed mechanism, it is expected that the ratio $E_1/E_2$, which is linked to the effective shear modulus $G_{\mathrm{eff}}$ of the real wing-box, is reduced in various orders of magnitude once the buckling is triggered. After this event, the results above show that the bending stiffness of the wing-box reduction will be negligible in comparison with the reduction in torsional stiffness. 

  \subsection{Parametric study results and discussion} \label{subsec:results_parametricStudy}

    In the present subsection, the variation of the beam mechanical properties for different geometric parameter values is shown. The beam geometry is characterized through the cross-sectional aspect ratio $B/H$, the thickness ratio $t_2/t_1$ and the slenderness ratio $L/B$. The effect of these parameters on the sectional properties, twist and bending stiffness, and flexural and twisting compliance are shown. Additionally, the variance of the stiffness ratio $E_1/E_2$ is also included in the analysis.

    The influence of the cross-sectional aspect ratio $B/H$ on the torsional stiffness $G I_t$, the shear centre position $y_{\mathrm{SC}}$ and the flexural stiffness $E I_y$ is shown in Figures \ref{fig:GIt-E1overE2-BoverH}, \ref{fig:SC-E1overE2-BoverH} and \ref{fig:EIy-E1overE2-BoverH}, respectively. On its side, the effect of thickness ratio $t_2/t_1$ on the same three beam parameters is shown in Figures \ref{fig:GIt-E1overE2-t2overt1}, \ref{fig:SC-E1overE2-t2overt1} and \ref{fig:EIy-E1overE2-t2overt1}.

    In Figure \ref{fig:GIt-E1overE2-BoverH}, it can be seen that a maximum torsional stiffness appears for $B/H = 1$ when $E_1/E_2 = 1$. This can be explained because, as it is also shown in \cite{Raither2013a}, the closer the torsional stiffness to the doubly symmetric case, the higher its torsional stiffness. However, when $E_1/E_2 > 10$, the maximum torsional stiffness is shown to appear for $B/H > 1$. A similar conclusion can be extract when analyzing the Figure \ref{fig:GIt-E1overE2-t2overt1}, that shows the influence of the thickness ratio $t_2/t_1$ on the torsional stiffness $G I_t$.

    In Figure \ref{fig:SC-E1overE2-BoverH} shows that for values $E_2 \ll E_1$, the shear centre position $y_{\mathrm{SC}}$ is approximately constant for $B/H$ variations. In this context, the beam approximates its behavior as if it has an open profile section. However, as the value of $E_1/E_2$ decreases, the influence of the ratio $B/H$ increases showing a bigger influence of the web where the Young's modulus $E_2$ applies. On the other hand, Figure \ref{fig:SC-E1overE2-t2overt1} shows that the bigger the thickness ratio $t_2/t_1$ is, the closer that the shear centre $y_{\mathrm{SC}}$ will be to the vertical axis of symmetry. However, for $E_2 \ll E_1$ the influence of the thickness ratio $t_2/t_1$ is reduced.

    In Figure \ref{fig:EIy-E1overE2-BoverH} it can be seen that the flexural stiffness $E I_y$ decreases when the cross-sectional aspect ratio $B/H$ increases. This is explained because the second moment of area $I_y$ is reduced when $B/H$ increases as the coordinate $y$ of the points in the section is reduced as well. Similarly, it can be seen in Figure \ref{fig:EIy-E1overE2-t2overt1} that when the thickness ratio $t_2/t_1$ increases, the value of $E I_y$ decreases. In both cases, the flexural stiffness $E I_y$ is little affected by variations of the stiffness ratio $E_1/E_2$. This result matches what was already shown in Subsection \ref{subsec:bendingTwistCoupling_results_model}.

    % Effect on deflection and torsional compliance
    Additionally, the effect of the cross-sectional aspect ratio $B/H$ on the deflection and torsional compliances is shown on Figures \ref{fig:woverQ-E1overE2-BoverH} and \ref{fig:phioverQ-E1overE2-BoverH}, respectively. The corresponding plots when analyzing the effect of the thickness ratio $t_2/t_1$ on the deflection and torsional compliance are shown on Figures \ref{fig:woverQ-E1overE2-t2overt1} and \ref{fig:phioverQ-E1overE2-t2overt1}, respectively. The beam torsional compliance is expressed as fraction of the twist at the tip divided by the vertical force applied, that is $|\phi_{\mathrm{tip}}| / Q$, while the beam deflection compliance is expressed as fraction of the maximum vertical displacement at the tip divided by the vertical force applied, that is $w_{\mathrm{0,tip}} / Q$.

    From the analysis of the torsional compliance $|\phi_{\mathrm{tip}}| / Q$ as a function of the thickness ratio $t_2/t_1$ in Figure \ref{fig:woverQ-E1overE2-t2overt1}, it can be seen that increments in $t_2/t_1$ amplifies the effects of variations of the stiffness ratio $E_1/E_2$ on the deflection compliance $w_{\mathrm{0,tip}} / Q$. On the other hand, Figure \ref{fig:woverQ-E1overE2-BoverH} shows that the deflection compliance $w_{\mathrm{0,tip}} / Q$ increases when the cross-sectional aspect ratio $B/H$ increases and the value of $B/H$ does not alter the effect of variations of $E_1/E_2$ in the deflection compliance, effect that remains small.

    From Figure \ref{fig:phioverQ-E1overE2-t2overt1}, it can be seen that the there is not variation in the torsional compliance for values of the thickness ratio $t_2/t_1 > 1$, but for $t_2/t_1 < 1$, the torsional compliance increases considerably. This result shows again something similar at what it was shown in Subsection \ref{subsec:bendingTwistCoupling_results_model}, how the modification of shear stiffness in the second web has a considerable effect on the torsional stiffness. On the other hand, Figure \ref{fig:phioverQ-E1overE2-BoverH} shows small effect of the variation of the cross-sectional aspect ratio $B/H$ on the torsional compliance.

    The effect of the slenderness ratio $L/B$ on the deflection and torsional compliances is shown in Figures \ref{fig:woverQ-E1overE2-LoverB} and \ref{fig:phioverQ-E1overE2-LoverB}, respectively. These two figures show how the deflection and torsional compliance increases for increasing values of $L/B$. Also, Figure \ref{fig:woverQ-E1overE2-LoverB} shows again the small effect of $E_1/E_2$ in the deflection compliance while Figure \ref{fig:phioverQ-E1overE2-LoverB} shows how the torsional compliance is considerably affected by variations of $E_1/E_2$.

    %Figures variation of B/H
    \begin{figure}[!htpb] %G I_t versus B/H
      \centering
      \includegraphics[width=0.8 \textwidth]{../../analytical/figures/GIt-E1overE2-BoverH}
      \caption[Influence of the cross-sectional aspect ratio $B/H$ on the torsional stiffness $GI_t$]{Influence of the cross-sectional aspect ratio $B/H$ on the torsional stiffness $GI_t$, shown for various values of the stiffness ratio $E_1/E_2$ ranging from $10^0$ to $10^3$. }\label{fig:GIt-E1overE2-BoverH}
    \end{figure}

    \begin{figure}[!htpb] %Shear centre versus B/H
      \centering
      \includegraphics[width=0.8 \textwidth]{../../analytical/figures/SC-E1overE2-BoverH}
      \caption[Influence of the cross-sectional aspect ratio $B/H$ on the dimensionless shear centre position $y_{\mathrm{SC}}/B$]{Influence of the cross-sectional aspect ratio $B/H$ on the dimensionless shear centre position $y_{\mathrm{SC}}/B$, shown for various values of the stiffness ratio $E_1/E_2$ ranging from $10^0$ to $10^3$. }\label{fig:SC-E1overE2-BoverH}
    \end{figure}

    \begin{figure}[!htpb] %E I_y = \Phi_y versus B/H
      \centering
      \includegraphics[width=0.8 \textwidth]{../../analytical/figures/EIy-E1overE2-BoverH}
      \caption[Influence of the cross-sectional aspect ratio $B/H$ on the flexural stiffness $EI_y$]{Influence of the cross-sectional aspect ratio $B/H$ on the flexural stiffness $EI_y = \Phi_y$, shown for various values of the stiffness ratio $E_1/E_2$ ranging from $10^0$ to $10^3$. }\label{fig:EIy-E1overE2-BoverH}
    \end{figure}

    %%%% Figures variation of t2/t1
    \begin{figure}[!htpb] %G I_t versus t2/t1
      \centering
      \includegraphics[width=0.8 \textwidth]{../../analytical/figures/GIt-E1overE2-t2overt1}
      \caption[Influence of the wall thickness ratio $t_2/t_1$ on the torsional stiffness $GI_t$]{Influence of the wall thickness ratio $t_2/t_1$ on the torsional stiffness $GI_t$, shown for various values of the stiffness ratio $E_1/E_2$ ranging from $10^0$ to $10^3$. }\label{fig:GIt-E1overE2-t2overt1}
    \end{figure}

    \begin{figure}[!htpb] %Shear centre versus t2/t1
      \centering
      \includegraphics[width=0.8 \textwidth]{../../analytical/figures/SC-E1overE2-t2overt1}
      \caption[Influence of the wall thickness ratio $t_2/t_1$ on the dimensionless shear centre position $y_{\mathrm{SC}}/B$]{Influence of the wall thickness ratio $t_2/t_1$ on the dimensionless shear centre position $y_{\mathrm{SC}}/B$, shown for various values of the stiffness ratio $E_1/E_2$ ranging from $10^0$ to $10^3$. }\label{fig:SC-E1overE2-t2overt1}
    \end{figure}

    \begin{figure}[!htpb] %E I_y = \Phi_y versus t2/t1
      \centering
      \includegraphics[width=0.8 \textwidth]{../../analytical/figures/EIy-E1overE2-t2overt1}
      \caption[Influence of the wall thickness ratio $t_2/t_1$ on the flexural stiffness $EI_y$]{Influence of the wall thickness ratio $t_2/t_1$ on the flexural stiffness $EI_y = \Phi_y$, shown for various values of the stiffness ratio $E_1/E_2$ ranging from $10^0$ to $10^3$. }\label{fig:EIy-E1overE2-t2overt1}
    \end{figure}

    % Displacements
    \begin{figure}[!htpb] %w_0,tip / Q versus B/H, deflection compliance
      \centering
      \includegraphics[width=0.8 \textwidth]{../../analytical/figures/woverQ-E1overE2-BoverH}
      \caption[Influence of the cross-sectional aspect ratio $B/H$ on the deflection compliance]{Influence of the cross-sectional aspect ratio $B/H$ on the deflection compliance $w_{\mathrm{0,tip}} / Q$, shown for various values of the stiffness ratio $E_1/E_2$ ranging from $10^0$ to $10^3$. }\label{fig:woverQ-E1overE2-BoverH}
    \end{figure}

    \begin{figure}[!htpb] %\phi_tip / Q versus B/H, torsional compliance
      \centering
      \includegraphics[width=0.8 \textwidth]{../../analytical/figures/phioverQ-E1overE2-BoverH}
      \caption[Influence of the cross-sectional aspect ratio $B/H$ on the torsional compliance]{Influence of the cross-sectional aspect ratio $B/H$ on the torsional compliance $|\phi_{\mathrm{tip}}| / Q$, shown for various values of the stiffness ratio $E_1/E_2$ ranging from $10^0$ to $10^3$. }\label{fig:phioverQ-E1overE2-BoverH}
    \end{figure}

    \begin{figure}[!htpb] %w_0,tip / Q versus t2/t1, deflection compliance
      \centering
      \includegraphics[width=0.8 \textwidth]{../../analytical/figures/woverQ-E1overE2-t2overt1}
      \caption[Influence of the thickness ratio $t2/t1$ on the deflection compliance]{Influence of the thickness ratio $t2/t1$ on the deflection compliance $w_{\mathrm{0,tip}} / Q$, shown for various values of the stiffness ratio $E_1/E_2$ ranging from $10^0$ to $10^3$. }\label{fig:woverQ-E1overE2-t2overt1}
    \end{figure}

    \begin{figure}[!htpb] %\phi_tip / Q versus t2/t1, torsional compliance
      \centering
      \includegraphics[width=0.8 \textwidth]{../../analytical/figures/phioverQ-E1overE2-t2overt1}
      \caption[Influence of the thickness ratio $t2/t1$ on the torsional compliance]{Influence of the thickness ratio $t2/t1$ on the torsional compliance $|\phi_{\mathrm{tip}}| / Q$, shown for various values of the stiffness ratio $E_1/E_2$ ranging from $10^0$ to $10^3$. }\label{fig:phioverQ-E1overE2-t2overt1}
    \end{figure}

    %%%% Figures variation of L / B
    \begin{figure}[!htpb] %w_0,tip / Q versus L/B, deflection compliance
      \centering
      \includegraphics[width=0.8 \textwidth]{../../analytical/figures/woverQ-E1overE2-LoverB}
      \caption[Influence of the slenderness ratio $L/B$ on the deflection compliance]{Influence of the slenderness ratio $L/B$ on the deflection compliance $w_{\mathrm{0,tip}} / Q$, shown for various values of the stiffness ratio $E_1/E_2$ ranging from $10^0$ to $10^3$. }\label{fig:woverQ-E1overE2-LoverB}
    \end{figure}

    \begin{figure}[!htpb] %\phi_tip / Q versus L/B, torsional compliance
      \centering
      \includegraphics[width=0.8 \textwidth]{../../analytical/figures/phioverQ-E1overE2-LoverB}
      \caption[Influence of the slenderness ratio $L/B$ on the torsional compliance]{Influence of the slenderness ratio $L/B$ on the torsional compliance $|\phi_{\mathrm{tip}}| / Q$, shown for various values of the stiffness ratio $E_1/E_2$ ranging from $10^0$ to $10^3$. }\label{fig:phioverQ-E1overE2-LoverB}
    \end{figure}

\clearpage
\section{Computational model analysis} \label{sec:computationalModelAnalysis_results_model}

  In the present section, results from the analysis of the computational model are presented. Different aspects of the model are evaluated and design decisions are justified. The analysis is going to consider the modeling of the lattice nodes rigid behavior, the influence of the mesh in the simulation convergence, the ribs addition and the nonlinearities that appear in the problem.

  % \subsection{Connection between the chiral lattice and the wing-box skin} \label{subsec:connection_results_model}
  %
  %   The modeling of the lattice nodes through either the coupling through a reference point and through tyre was shown to
  %
  %For the parametric study that was performed this reason, it was decided not to use a model where the connection between the chiral lattice and the wing-box skin is modeled following the approach exposed in the Subsection \ref{subsec:connections_computationalModel}. Instead, a more rough connection is used, as shown in Figure \ref{fig:connectionSimple}.
  %
  %    \begin{figure}[!htpb]
  %      \centering
  %      \includegraphics[width=0.8 \textwidth]{result-model/connectionSimple}
  %      \caption[Connection between the chiral lattice and the wing-box for the parametric study]{Connection between the chiral lattice and the wing-box for the parametric study.}\label{fig:connectionSimple}
  %    \end{figure}
  %
  \subsection{Rigid body modeling} \label{subsec:rigidBody_results_model} %To be reviewed
    % -> Rotation of the nodes was like 20\% inferior with the coupling
    
    Two different approaches were followed to model the rigid body behavior of the lattice nodes, as it been already explained in Subsection \ref{subsec:latticeNodesRigid_Parametrization}. The first model approach considers the creation of a coupling condition between a reference point positioned at the centre of the lattice node cylinder and the mesh nodes found on the faces of the lattice node. The second approach consisted on an additional part that is placed in the middle of the cylinder to added stiffness to the element. In this section, the results obtained for each modeling approach are compared.

    If Figure \ref{fig:rigid-modeling}, the nonlinear and linear responses in wing-box twist obtained for each modeling approach are compared. It can be seen that when the tyre approach is used, the collapse of the structure occurred at a smaller load than if the approach with the coupling implementation is used.

    Also, the analysis of the displacements on results model showed that, when the tyre rigid body behavior is modeled using the tyre part, the rotation of the lattice nodes around its own axis is approximately 20\% higher than when the modeling is done using the coupling constraint. This rotation should be free and only be constrained by the connection to the ligaments. Therefore, it is concluded that the use of the tyre part is a more realistic approach to model the rigid body behavior of the lattice nodes.

    \begin{figure}[!htpb]
      \centering
      \includegraphics[width=0.8 \textwidth]{result-model/rigid-modeling}
      \caption[Displacement-force curve obtained using different approaches to model the rigid lattice node]{Displacement-force curve obtained using different approaches to model the rigid lattice node. }\label{fig:rigid-modeling}
    \end{figure}

  \clearpage
  \subsection{Load introduction method} \label{subsec:load_results_model}

    As shown in Subsection \ref{subsec:load_computationalModel}, there are multiple ways in which the load can be introduced into the structure. It was decided to locate the load introduction points at the upper flange of the ribs in an attempt of replicate how the load introduction would be done in a future manufactured demonstrator. In this subsection, the difference of results obtained when modifying the load introduction method are shown. 

    Firstly, the option of distributing a same load magnitude over the total number of ribs, three in the baseline configuration, is compared with the option of concentrating all the load on the upper flange of the tip rib. In Figure \ref{fig:forceDisplacement-distributedLoad700N} the curve displacement-force for these two different cases is shown. It can be seen how when the load is distributed over a number of points and it is not concentrated on a single point, a delay in the onset of the structure collapse occurs. This results is correlated with visual inspection of the solution model in Abaqus. The corresponding plot for a distributed load is shown in Figure \ref{fig:distributedLoad700N}. Here it can be seen how buckling has started to occur on the ligaments located close to the ribs, where the load is being introduced. In a further stage, if the load would increase, buckling would start to occur at the root of the wing-box just like it happens for a single load introduction point.

    Secondly, in Figure \ref{fig:forceDisplacement-loadZ0coma8} a comparison of different load position points in the chordwise direction is shown. The plot shows the displacement-force curve for four values of the variable $z/W_{\mathrm{box}}$. A smaller value of $z/W_{\mathrm{box}}$ represents a load introduction point located closer to the chiral lattice. It can be seen that buckling is not triggered for values of $z/W_{\mathrm{box}}$ equal to 0.6 and 0.8. This is explains because the further the introduction point is from the chiral lattice, the less shear strain is transmitted to the chiral lattice and the later that the onset of the elastic instabilities will occur. 

    Since the distribution of the load in more that one rib only delays the onset of the instabilities, it is decided to use a single load introduction point for the study performed on the baseline configuration, in a attempt to simplify the model. Also, it was decided to locate the load introduction point at $z/W_{\mathrm{box}} = 0.5$ to keep the symmetry.

    \begin{figure}[!htpb]
      \centering
      \includegraphics[width=0.8 \textwidth]{../figures/result-model/forceDisplacement-distributedLoad700N}
      \caption[Displacement-force curve showing the model response using different load introduction methods]{Displacement-force curve showing the model response using different load introduction methods. The label ``singleForceOnLastRib\_upper'' corresponds to the case of using one single point for the load introduction. This point is located at the upper flange of the tip rib. The label ``linForceInnerRibs\_upper'' corresponds to the case of introducing load on the upper flange of all the available ribs, which are three for the baseline configuration, two in the inner part of the wing-box and one at the tip. It can be see that the concentration of the load accelerates the onset of the structure collapse.}
      \label{fig:forceDisplacement-distributedLoad700N}
    \end{figure}

    \begin{figure}[!htpb]
      \centering
      \includegraphics[width=0.8 \textwidth]{../figures/result-model/distributedLoad700N}
      \caption[Deformed state of the model for a distributed load introduction method]{Deformed state of the model for a distributed load introduction method. It can be seen how since the buckling is starting to occur at the ligaments located close to the ribs, the onset of the elastic instabilities is delayed.}
      \label{fig:distributedLoad700N}
    \end{figure}

    \begin{figure}[!htpb]
      \centering
      \includegraphics[width=0.8 \textwidth]{../figures/result-model/forceDisplacement-loadZ0coma8}
      \caption[Displacement-force curve showing the model response for different load introduction points $z/W_{\mathrm{box}}$]{Displacement-force curve showing the model response for different load introduction points $z/W_{\mathrm{box}}$. A smaller value of $z/W_{\mathrm{box}}$ represents a load introduction point located closer to the chiral lattice. The further the introduction point is from the chiral lattice, the less shear strain is transmitted to the chiral lattice and the later that the onset of the elastic instabilities will occur.}
      \label{fig:forceDisplacement-loadZ0coma8}
    \end{figure}

  \clearpage
  \subsection{Mesh} \label{subsec:mesh_results_model}

    The model was build using cell shell elements as the fundamental constituting part. The thickness is assigned in the perpendicular direction, as it was shown in Figure \ref{fig:shellElement}.

    This type of element is a 2D element that it was used to build 3D structures. This kind of procedure may incur some distortion in the mesh elements due to shell elements intersecting in the same line at different angles.

    For the designed model, this situation occurred at certain points of the chiral lattice. It can be seen in Figure \ref{fig:meshDistorted} that the distorted elements appear at two different positions mainly. Firstly, at the plane where the two ligaments with different curvature join. At this point, the sharp angles that appear in between the part geometrical lines induce the appearance of tetrahedral distorted mesh elements. The second typical location for appearance of distorted elements is along the curve where the lattice nodes and the curved ligaments join.

    \begin{figure}[!htpb]
      \centering
      \includegraphics[width=0.8 \textwidth]{result-model/meshDistorted}
      \caption[Distorted mesh elements in the model]{Distorted mesh elements in the model. The number of distorted elements was found to be crucial for the simulation convergence.}\label{fig:meshDistorted}
    \end{figure}

    It was seen that the number of distorted elements had a significant effect in the simulation convergence evolution. For a high number of distorted elements, the simulation could not go further from the first step. No attempts to locally modify the mesh at the mentioned locations were made, instead, it was found that modifying the global mesh size gave enough control over the number of distorted elements to be able to overcome this limitation. The bigger the mesh size in the area, the less distorted elements appear after completing the meshing operations.

  \clearpage
  \subsection{Nonlinear problem and automatic stabilization} \label{subsec:nonlinear_results_model} %OK

    For the case under study, nonlinear simulations will be carried out as is expected to find a nonlinear displacement-force curve as a result of the analysis. In Abaqus, to execute nonlinear simulation involves the following, as shown in \cite{Abaqus}:

    \begin{itemize}
      \item a combination of incremental and iterative procedures;
      \item using the Newton method to solve the nonlinear equations;
      \item determining convergence;
      \item defining loads as a function of time; and
      \item choosing suitable time increments automatically.
    \end{itemize}

    Therefore, Abaqus breaks the step where the load is applied into increments. The software will automatically choose the size of each of the increments based on the convergence evolution of previous increments.

    Also, nonlinear static problems may become unstable. One of the possible sources of such instabilities is buckling. A model where buckling appears locally may not be resoluble using general solution methods. For this kind of cases, it becomes necessary to either solve the problem dynamically or with the aid of artificial damping.

    Since the above situation represents what it is expected to be found in the model response, a constant artificial damping factor is used throughout the whole step to account for the appearance of local instabilities.

    Automatic stabilization with a constant damping factor implies that viscous forces of the form:
    \begin{equation}
      F_v = c \mat{M} v
    \end{equation}
    are added to the global equilibrium equations:
    \begin{equation}
      P - I - F_v = 0,
    \end{equation}
    %
    where $I$ represents the internal forces, $P$ the external forces, $\mat{M}$ is the artificial mass matrix calculated with unity density, $c$ is the defined damping factor, $v = \Delta u / \Delta t$ is the vector of nodal velocities, and $\Delta t$ is the increment of simulation time.

    The final value of $c$ that is going to be used during the simulations is chosen after performing a small parametric study of the different possibilities. As a result, the plot shown in Figure \ref{fig:forceDisplacement-damp} is produced. This plot represents the evolution of the twist at the tip as the load is increased step by step during the nonlinear simulation. As it will be explained in Section \ref{sec:generalResponseCharact_results_sim}, the use of automatic stabilization becomes necessary to capture the dynamics that involve buckling on the chiral ligaments and the ultimate collapse of the structure.

    \begin{figure}[!htpb]
      \centering
      \includegraphics[width=0.8 \textwidth]{../figures/result-model/forceDisplacement-damp}
      \caption[Displacement-force curve for various values of constant artificial damping factor]{Displacement-force curve for various values of constant artificial damping factor.}\label{fig:forceDisplacement-damp}
    \end{figure}

    As it can be seen in the mentioned figure, all the different values of the damping factors success to capture the rapidly change in tip twist that occurs for fractions of load applied $>60\%$. However, special care needs to be taken in order to ensure that the inclusion of artificial damping factor is not leading to inaccurate results due to over-damping of the structure. This can be done by comparing the fraction of the static energy that it is dissipated compared to the external work that its put into the system. This is done for a values of $c =$\notcien{2}{-5}, $c =$\notcien{2}{-8} and $c =$\notcien{2}{-9} in Figures \ref{fig:energy_damp-5}, \ref{fig:energy_damp-8} and \ref{fig:energy_damp-9}, respectively. In these plots, the moment where the structure collapses due to the buckling phenomena occurring on the chiral ligaments can be seen as a sudden change in the slope of both curves. Here it can be seen that for the case of $c =$\notcien{2}{-5}, the slope of the curve showing the energy dissipated through artificial stabilization is positive after the onset of buckling, which is a sign of over-damping in this region. On the other hand, for the case of $c =$\notcien{2}{-8}, the slope remains of the curve remains zero. Finally, for the case of $c =$\notcien{2}{-9}, the slope also remains zero and the final value for the energy dissipated is smaller than for $c =$\notcien{2}{-8}.

    \begin{figure}[!htpb]
      \centering
      \includegraphics[width=0.8 \textwidth]{../figures/result-model/energy_damp-5}
      \caption[External work and static dissipation for a damping factor equal to \notcien{2}{-5}]{External work and static dissipation for a damping factor equal to \notcien{2}{-5}. The positive slope of the curve showing the energy used in the static dissipation is a sign of over-damping.}\label{fig:energy_damp-5}
    \end{figure}

    \begin{figure}[!htpb]
      \centering
      \includegraphics[width=0.8 \textwidth]{../figures/result-model/energy_damp-8}
      \caption[External work and static dissipation for a damping factor equal to \notcien{2}{-8}]{External work and static dissipation for a damping factor equal to \notcien{2}{-8}. After the structure collapse the static dissipation energy remains constant and small compared with the external work introduced into the system.}\label{fig:energy_damp-8}
    \end{figure}

    \begin{figure}[!htpb]
      \centering
      \includegraphics[width=0.8 \textwidth]{../figures/result-model/energy_damp-9}
      \caption[External work and static dissipation for a damping factor equal to \notcien{2}{-9}]{External work and static dissipation for a damping factor equal to \notcien{2}{-9}. After the structure collapse the static dissipation energy remains constant and negligible compared with the external work introduced into the system.}\label{fig:energy_damp-9}
    \end{figure}

    Finally, since the objective is to capture the dynamics that involve buckling on the chiral ligaments and the ultimate collapse of the structure, keeping the energy dissipated a small as possible fraction of the external work, it is decided to use constant damping factor $c =$\notcien{2}{-9} for the simulations perform ahead. 

  \clearpage
  \subsection{Ribs} \label{subsec:ribs_results_model}
    %
    %Solution characterization
    % -> For the normal case
    %     - \ref{fig:normalCaseNoDampNoInnerRibs_800N}
    %     - the initial buckling ocurrs far from the root
    %     - UR1 of ~0.3 deg at the beam tip
    %
    % -> Introduccing damping
    %     - \ref{fig:normalCaseNoDampNoInnerRibs_800N}
    %     - Really big local deformations appear on the wing-box upper surface
    %     - UR1 of one order of magnitude higher, 3deg at the tip
    %     - Introduccing inner ribs becomes necessary to achieve greater twist with reduced local deformation
    %     - There is a point at which the structure collapses
    %     - \ref{fig:normalCaseDampNoInnerRibs}
    %     - See plot that compares the external work introduced in the system and the static energy dissipated through artificial stabilization. \ref{fig:normalCaseNoDampNoInnerRibs_800N_stabilizationPlot}
    %
    % -> Introducing inner ribs but not damping
    %     - \ref{fig:normalCaseNoDamp2InnerRibs_800N}
    %     - Buckling again appears at the front at it is very little
    %     - The simulation stops really soon, only tau/T = 0.134 achieved for the current plot
    % -> Introducing damping and inner ribs
    %     - Twist at the tip is now of about 0.68deg
    %     - The energy plot for this is \ref{fig:normalCaseDamp2InnerRibs_800N_stabilizationPLot}
    %     - The deformation plot shows now how buclking start at the lattice located close to the inner rib more close to the root, in a positive x from the position of the rib. The buckling then moves a group of lattices at the root, in the upper position of the lattice.
    %     - Some local deformation still ocurrs at the section of the wing-box skin located in between the root at the first inner rib.
    %     - However, the inner ribs help the upper skin of the wing-box to remain more flat than before.

    %First approach: no ribs, no damp
    Initially, the model did not incorporate inner ribs and the ribs located at the tip and the root had an open profile. In Figure \ref{fig:closeOfRib-800N}, it can be seen an example of the response seen for this type of configuration.

    \begin{figure}[!htpb]
      \centering
      \includegraphics[width=0.8 \textwidth]{../figures/result-model/closeOfRib-800N}
      \caption[Vertical displacement $v$ at the tip rib]{Vertical displacement $v$ at the tip rib. The color contour shows those mesh nodes located on the upper flange of the rib have a higher $v$, therefore showing how the rib is closing under the prescribed load (800 N)}\label{fig:closeOfRib-800N}
    \end{figure}

    For this reason, it was decided to use ribs with a close profile. The simulations then provided a solution like the one shown in Figure \ref{fig:normalCaseNoDampNoInnerRibs_800N}. For this case, the initial buckling occurs in ligaments located far from the root. The twist of the beam, measured as the angular rotation $UR_1$ around the $x$ axis is of ~0.3 deg. The prescribed load was -800 N and the simulation converged to the 0.95\% of the prescribed load.

    \begin{figure}[!htpb]
      \centering
      \includegraphics[width=0.8 \textwidth]{../figures/result-model/normalCaseNoDampNoInnerRibs_800N}
      \caption[Model response without the use of inner ribs nor automatic stabilization]{Model response without the use of inner ribs nor automatic stabilization.}\label{fig:normalCaseNoDampNoInnerRibs_800N}
    \end{figure}

    In order to investigate further deformations of the ligaments, it was decided to carry out simulations that incorporate automation stabilization through artificial damping artificial damping factor, as it was explained in Subsection \ref{subsec:nonlinear_results_model}. After this, the results showed a deformation like the one shown in Figure \ref{fig:normalCaseDampNoInnerRibs}. This figure shows how big local deformations appear on the wing-box upper skin for this case. Also, it can be seen that the buckling phenomena has moved backwards to the ligaments close to the root.

    \begin{figure}[!htpb]
      \centering
      \includegraphics[width=0.8 \textwidth]{../figures/result-model/normalCaseDampNoInnerRibs}
      \caption[Model response without the use of inner ribs with automatic stabilization]{Model response without the use of inner ribs with automatic stabilization.}\label{fig:normalCaseDampNoInnerRibs}
    \end{figure}

    %Inner ribs, no damp
    In order to reduce the local deformations occurring on the wing-box, a pair of inner ribs as described in Subsection \ref{subsec:parametrization_Model} were added to the model. This element added stiffness to the structure in bending. Now, the response of the model was shown to be like the one represented in Figure \ref{fig:normalCaseNoDamp2InnerRibs}. Here it can be seen that the ligaments that start to buckle are located at the same position as they were in the response of the model that did not incorporate inner ribs seen in Figure \ref{fig:normalCaseNoDampNoInnerRibs_800N}. However, now the degree of deformation has decreased due to the stiffness added to the structure as a result of the inner ribs addition.

    \begin{figure}[!htpb]
      \centering
      \includegraphics[width=0.8 \textwidth]{../figures/result-model/normalCaseNoDamp2InnerRibs_800N}
      \caption[Model response incorporating inner ribs and without automatic stabilization]{Model response incorporating inner ribs and without automatic stabilization}
      \label{fig:normalCaseNoDamp2InnerRibs}
    \end{figure}

    %Now damp and inner ribs
    However, for this last case, the simulation was only able to converge up to 13\% of the prescribed load. In other to progress further in the analysis, the use of automatic stabilization becomes necessary again. For this reason, in the final configuration, automatic stabilization through constant damping factor will be used together with the inner ribs. A description of the response of the structure for this last case is presented on next chapter.