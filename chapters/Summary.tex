\chapter{Conclusion and Outlook} \label{chap:summary}
%
%Outlook:
% - To improve the mesh characteristics (to avoid distorted elements)
%
%Summary:

This thesis presents a novel passive mechanism of wing twist morphing achieved through the control of the bending-twist coupling of the wing-box. The system incorporates a variable-stiffness spar in the wing-box that comprises a lattice of chiral elements. In the ligaments of these elements, elastic instabilities are intentionally induced and thus the effective shear modulus of the adaptive spar is modified. This adaptation provokes the wing-box section shear centre shifting and the consequent modification in wing-box torsional stiffness. This, ultimately induces a twist deformation in wing-box that has been passively activated exploiting local elastic instabilities.

An analytical model of the wing-box is developed using an ideal beam configuration with variable shear modulus in one of the webs. The effect of variations of this magnitude on the torsional stiffness, the flexural stiffness, the shear centre position and the bending and twisting deformations of the wing-box are studied. Also, the effect of variations of the geometry in these magnitudes is analysed. Results anticipate that, once th buckling-induced reduction in shear modulus in the spar is activated, the reduction in flexural stiffness for the wing-box is negligible compared with the reduction in torsional stiffness. 

A computational model of the whole assembly is built next. It is designed in a fully parameterized format using Python scripting. This model is constituted of a wing-box in C-profile, a lattice of chiral elements and a tunable number of ribs that provide additional transversal stiffness and avoid local deformations in the wing-box skin. Different design options are considered, such as variations in the geometry, the load, the mesh and the number of ribs. An analysis of this model is performed in order to obtain a baseline configuration to posteriorly execute parametric studies. One of the aspects considered is this analysis is the modeling of the lattice nodes rigid body behavior. Results shows that the best modeling approach is to add an additional rigid part to provide supplementary stiffness. The addition of ribs in the middle of wing-box length is considered necessary to provide additional stiffness in the transversal direction and avoid undesired local deformations in the wing-box skin. Another consideration is the appearance of sources of instabilities such as buckling that require the inclusion of artificial damping factor into the simulation. 

Nonlinear simulations are carried out using this model and incorporating artificial dissipation through constant damping factor. Special attention needs to be taken so that the inclusion of artificial damping factor is not leading to inaccurate results due to over-damping of the structure. The onset of the buckling phenomena and its evolution in the chiral lattice is characterized next. Results show that the collapse of the structure occurs when severe buckling appears on the ligaments of the chiral elements located at the root of the wing-box. This event produces a sudden reduction on the torsional stiffness of the structure and an increase in the twist deformation observed at the wing-box tip. A second and more generalized buckling in the chiral lattice is observed to origin a second modification of the torsional stiffness for certain cases. For the baseline configuration of the model, the twist morphing of the wing-box is obtained to be equal to -1.25 degrees for the nonlinear simulation while the predicted twist for the linear simulation is -0.19 degrees.

Furthermore, a parametric study is performed on the computational model. Results show that considerable tailorability can be achieved through modifications of selected parameters. The parameter that shows to have a bigger influence in the onset and evolution of the elastic instabilities is the wing-box thickness. Increasing the wing-box thickness by 0.1 mm, provokes that the required force to induce the appearance of buckling at the root increases by 100 N. 

In further studies of this technology, it would be necessary to numerically analyse the system response for more realistic aerodynamic loads and also manufacture a demonstrator that could be used to experimentally test the feasibility of the proposed concept. The analysis of the time-bounded response of this mechanism is beyond the scope of this preliminary work, but such a investigation would be a crucial prior implementation of the system on a real lift generating structure

The proposed morphing has been shown to be capable of inducing global twist morphing of a wing-box exploiting local elastic instabilities. It has shown to be promising to be applied on realistic wing structures for which the torsional response is dominated by the wing-box. Possible applications may include load alleviation purposes on structures with high aspect ratio. 