\chapter{Conclusion and Outlook} \label{chap:summary}
%
%Outlook:
% - To improve the mesh characteristics (to avoid distorted elements)
%
%Summary:

This thesis presents a novel mechanism to achieve wing twist morphing through variations in the wing-box torsional stiffness. The method comprises a wing-box with a variable-stiffness spar which is designed as a lattice of chiral structures formed by interconnected ligaments and nodes. The onset of elastic instabilities in the chiral ligaments, provokes that the effective shear modulus for the whole chiral lattice is reduced. This enables the relocation of the wing-box section's shear centre and therefore the variation of the torsional moment acting on the wing-box. In this way, the onset of elastic instabilities in the adaptive spar triggers a twisting morphing deformation in the wing-box.

An analytical model of the wing-box is developed using a ideal beam configuration with variable shear modulus in one of the web. The effect of variations of this magnitude on the torsional stiffness, the flexural stiffness, the shear centre position and the bending and twisting deformations of the wing-box are studied. Also, the effect of variations of the geometry in these magnitudes is analyzed. Results anticipate that, once th buckling-induced reduction in shear modulus in the spar is activated, the reduction in flexural stiffness for the wing-box is negligible compared with the reduction in torsional stiffness. 

A computational model of the whole assembly is built next. It is designed in a fully parameterized format using Python scripting. Different design options are considered. Nonlinear simulations are carried out using this model and incorporating artificial dissipation through constant damping factor. Special attention needs to be taken to not originate a over-damping of the structure. The onset of the buckling phenomena and its evolution in the chiral lattice is characterized next. Results show that the collapse of the structure occurs when severe buckling appears on the ligaments of the chiral elements located at the root of the wing-box. This event produces a sudden reduction on the torsional stiffness of the structure and an increase in the twist deformation observed at the wing-box tip. A second and more generalized buckling in the chiral lattice is observed to origin a second modification of the torsional stiffness for certain cases. For the baseline configuration of the model, the twist morphing of the wing-box was obtained to be of XX.

Furthermore, a parametric study was performed on the computational model. Results show that considerable tailorability can be achieved through modifications of selected parameters. When considering the wing-box thickness, it was found that variations of XX mm made that the required force to provoke the onset of the structure collapse was increase in XX N...

COMMENTS ON THE POSSIBLE APPLICATIONS OF THIS TECHNOLOGY

