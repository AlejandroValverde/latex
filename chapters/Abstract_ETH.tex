\section*{Abstract}

This thesis presents a novel purely passive mechanism of twist morphing for application on aerospace structures, such as wings. During certain flight conditions such as gust encountering, the aircraft may experience critical loads that threaten the structural integrity. Such critical conditions can be prevented through modification of the lift distribution to achieve a reduction in the aerodynamic load is needed. Herein, the proposed method aims to control the bending-twist coupling of a wing-box that implements a variable-stiffness adaptive spar. The spar is designed through a chiral of chiral lattice. The particular chiral design enables to achieve the shear stiffness variability through inducing elastic instabilities locally in the chiral lattice for certain loading condition. The modification of the effective shear modulus in this element provokes the wing-box shear center shifting, a modification of the torsional stiffness and, as a result, a buckling-induced sectional twist in the wing-box. The mechanism therefore consist on a highly nonlinear variation of the wing twist passively triggered by the onset of buckling on the chiral structure. Initially, a simplified wing-box model is considered to show the feasibility of the concept. An analytical model of the wing-box is developed to provide insight of the effect of the shear modulus modification on the variable-stiffness spar mechanical properties. A fully parametrized computational model of the wing-box is also built to provide insight of the buckling phenomena and the nonlinear twist response of the structure. An extensive analysis of the influence of each of the main parameters in the buckling appearance and evolution, and in the final wing-box twist response is presented. Numeric results show that the snap-buckling instabilities indeed provide an effective method of changing the torsional stiffness of a wing-box and therefore affecting the twist and the lift distribution over the wing. In addition, the obtained results show that parameters such as the wing-box thickness or the ligament eccentricity provide tailorability of the mechanism activation and the response evolution. Departing from the simplified wing-box design, the concept is extended to realistic wing structures. BLABLABLA

In a further step, the wing-box model is embedded into a more complex model of a wing where realistic aerodynamic loads are applied on the wing skin. A weakly coupled static aeroelastic analysis is performed with this model. The effect wing-box torsional adaptation is therefore shown to reduce the bending moment at the root.

\vfill
\normalsize
\noindent
This thesis is submitted in partial (45\%) fulfillment of the requirements for the degree of Master of Science in Aerospace Dynamics at Cranfield University (UK).