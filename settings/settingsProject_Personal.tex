%Personal settings

%Important packages
\usepackage{graphicx}
\usepackage{subfigure}
\usepackage{longtable}
\usepackage{amsmath}


%Nomenclature settings
\usepackage{nomencl}
\makenomenclature
\renewcommand{\nomname}{List of symbols}
\newcommand{\nomunit}[1]{\renewcommand{\nomentryend}{\hspace*{\fill}[#1]}}
\newcommand{\mat}[1]{\mathrm{\textbf{#1}}}
\newcommand{\notcien}[2]{$#1\times10^{#2}$}
\renewcommand{\nomgroup}[1]{%
 \ifthenelse{\equal{#1}{A}}{\item[]}{% For roman letters
 \ifthenelse{\equal{#1}{B}}{\item[]}{% For greek letters
 \ifthenelse{\equal{#1}{C}}{\item \vspace{1em} \textbf{Abbreviations}}{% For abbreviations
 \ifthenelse{\equal{#1}{D}}{\item \vspace{1em} \textbf{Configurations}}{}}}}}

%Additional packages
\usepackage{ifthen}
\usepackage[final]{pdfpages} %Import pdf
\usepackage{xspace} %the xs­pace pack­age provides a single command that looks at what comes after it in the command stream, and decides whether to insert a space to replace one "eaten" by the TEX command decoder.

%bibliography
% \usepackage[square, numbers]{natbib}
% \usepackage[sorting=none, style=numeric]{biblatex}
% \addbibresource{projectBib}

\newcommand{\spaceBetweenEq}{6}

\newcommand{\ClearDoublePageOrNot}[1]
{
  \ifthenelse{\equal{#1}{electronicCopy}}{\clearpage}{}
  \ifthenelse{\equal{#1}{paperCopy}}{\clearpage{\pagestyle{empty}\cleardoublepage}}{}
}

\allowdisplaybreaks % - to allow page breaks inside equations

\graphicspath{ {figures/} } %Path where the figures are located

%Bibliography packages
\usepackage{hyperref} %Enabling re-directioning in the document for references

%Underline package
% \usepackage[normalem]{ulem}
% \useunder{\uline}{\ul}{}

%%%%%%%%%%%%%% HEADER AND FOOTER SETTINGS %%%%%%%%%%%%%%%%%%%

\usepackage{fancyhdr}

% To overhang the outside margin where the marginalnotes are printed, add both\marginparsep and \marginparwidth to \headwidth
% \addtolength{\headwidth}{\marginparsep} 
% \addtolength{\headwidth}{\marginparwidth}

\fancyhf{}% clear all header and footer fields
% \fancyhead[LE,RO]{\textbf{\thepage}}
\renewcommand{\chaptermark}[1]{\markboth{\chaptername\ \thechapter.\ #1}{}} %This gives: "Chapter 1: Results"
\renewcommand{\sectionmark}[1]{\markright{#1}} %Only gets the section name
\renewcommand{\headrulewidth}{0.4pt}
\renewcommand{\footrulewidth}{0.4pt}
%Headers and footers possitions
\ifthenelse{\equal{\controlClearPage}{paperCopy}}{%
	%Two-sided document options
	\fancyhead[LE,RO]{\rightmark}
	\fancyhead[RE,LO]{\textbf{\leftmark}}
	\fancyfoot[LE,RO]{\thepage}
	%Two-sided document options
	}{\ifthenelse{\equal{\controlClearPage}{electronicCopy}}{%
	%One-sided document options
	\fancyhead[RE,RO]{\rightmark}
	\fancyhead[LE,LO]{\textbf{\leftmark}}
	\fancyfoot[CE,CO]{- \thepage\ -}
	%One-sided document options
	}{}}

% Redefine plain style - This will apply to all the "contents" pages before the first chapter and to the first pages of each chapter
\fancypagestyle{plain}{%
\fancyhead{} % get rid of headers
\fancyfoot{} % clear all footer fields
%Footer position
\ifthenelse{\equal{\controlClearPage}{paperCopy}}{
	%Two-sided document options
	\fancyfoot[LE,RO]{\thepage}
	%Two-sided document options
	}{\ifthenelse{\equal{\controlClearPage}{electronicCopy}}{
	%One-sided document options
	\fancyfoot[CE,CO]{- \thepage\ -}
	%One-sided document options
	}{}}
\renewcommand{\footrulewidth}{0.4pt}
\renewcommand{\headrulewidth}{0pt} % and the line
\headheight=0.0cm
}

%%%%%%%%%%%%%%%%%%%%%%%%%%%%%%%%%%%%%%%%%%%%%%%%%%%%%%%%%%%%%%%%%%%

%New commands
\newcommand{\chiB}{$B_{\mathrm{chi}}$\xspace}
\newcommand{\chir}{$r_{\mathrm{chi}}$\xspace}
\newcommand{\chiL}{$L_{\mathrm{chi}}$\xspace}
\newcommand{\chit}{$t_{\mathrm{chi}}$\xspace}
\newcommand{\boxt}{$t_{\mathrm{box}}$\xspace}
\newcommand{\chie}{$\epsilon_{\mathrm{chi}}$\xspace}
\newcommand{\philin}{$\tilde{\phi}_{\mathrm{tip}}$\xspace}
\newcommand{\phinonlin}{$\phi_{\mathrm{tip}}$\xspace}